Plasticity, phenotypic diversity and phenotypic selection are all evolutionary properties that are found in this series of simulations, and evoke the multilevel selection model described by~\citet{jablonka2014evolution}. Among the three tested environmental fluctuations, the LF and SF simulations are showing the most similarities with this model. In ScF, the most successful evolutionary strategy seems to involve small genotypes, which could be favored for their ability to maximize the phenotypic impact of mutations. Furthermore, the inability of most ScF individuals to survive outside of the ecosystem resulting from their evolutionary history is reminiscent of the impossibility of saving species by the unique preservation of their DNA mentioned in \cite{jablonka2014evolution}: \say{You would have to reconstruct the community, and often these communities are very old, with historical memories that are stored in their epigenetic and behavioral systems. These are part of their ``identity,'' part of their stability. You cannot freeze these memories: they have to be maintained and transmitted through use, so you cannot reconstruct the communities from their component parts.} (\emph{ibid.},~p.~363). However, these experiments cannot determine whether the main indicator of differences between LF and SF one side and ScF on the other side is the duration of the environmental cycles or the number of different types of environments, and further investigation is needed.