\documentclass[letterpaper]{article}
\usepackage{natbib,alifeconf}
\usepackage{hyperref}


\usepackage[flushleft]{threeparttable}
\usepackage{subcaption}
\usepackage{booktabs}
\setcounter{tocdepth}{3}
\usepackage{array}

\title{Heterogeneous Cellular Automata Evolution in a Fluctuating Environment}
\author{First Author$^{1}$, Second Author$^{1,2}$, Third Author$^{1,2}$, Fourth Author$^{1,2}$ \and Fifth Author$^2$ \\
\mbox{}\\
$^1$First affiliation  \\
$^2$Second affiliation \\
corresponding@author.email}


\begin{document}
\maketitle

\begin{abstract}
The importance of environnemental fluctuations in the evolution of living beings by natural selection have been widely noted by biologists and have been linked to many important characteristics of the life as: modularity, plasticity, size of the genotype, mutation rate, learning or epigenetic adaptations. However, in artificial life simulations, environmental fluctuations are usually seen as a problem to be solved rather than an essential characteristic of evolution. We propose in this paper to use HetCA, an heterogeneous cellular automata characterized by its ability to generate open ended long-term evolution and evolutionary progress, to measure the impact of different forms of environmental fluctuations. Our results indicate that environnemental fluctuations induce mechanisms analogous to epigenetic adaptations in HetCA.
\end{abstract}





\section{Introduction}
If early population genetics theory assumed the environment to be constant, since Richards Levins works \cite{levins1968evolution} in the 70 up to Evolution in Four Dimensions by Eva Jablonka and al. \cite{jablonka2014evolution} the importance of environnemental fluctuations in the evolution of living beings by natural selection have been widely noted. Through those works and others, environmental fluctuations have been linked to many important questions about the mechanisms of evolution such as: modularity, plasticity, size of the genotype, mutation rate and through this the evolvability. Recently Jablonka\cite{jablonka2014evolution} have developed the idea that epigenetic mechanisms controlling the rate of changes in certain genes have evolved to counter frequent environmental changes. 
Environmental changes can include introduction of new predator or new potencial food, radical environment modifications such as climate change inducing environmental stress but also cyclic changes such as daily cycle of light and darkness, seasons...  

In \cite{lachmann1996inheritance} Lachemann model the consequences of such cycling variations on phenotypical inheritances. Their model predict that, when those cycles are longer than the reproductive cycle but relatively short heritable variations produced by non-DNA inheritance systems are likely to be observed.

Many work in artificial life approach the issue of environmental variations especially in evolutionary robotics\cite{floreano2000evolutionary}, but the greater part are interested mainly in the spacial variations, or considering them as a problem to solve.

This paper is organized as follows. Section 2 highlights consequences of variable environment in biological evolution. In Section 3, we explain the mecanisms of HetCA simulation. The implementation of environmental fluctuation in HetCA  is then explained in Section 4. Section 5 details the computer experimental setup while we reports experimental results in Section 6. We discuss the implications of these results in Section 7. Finally, Section 8 concludes the paper.


\section{Background}
In \cite{lipson2002origin} Lipson demonstrated a correlation between the modularity and the rate of change of the environment resources. While in \cite{yu2007program} Yu used  observed populations exploit neutrality to cope with environmental fluctuations and therefore evolve some sort of evolvability under 2 alternating objective functions. Both simulations used Genetic Programming (GP) and fitness explicit functions.

\section{HetCA}
HetCA is based on classical two-dimensional CA, to which it added the following features : Cells have "age", "decay", and "quiescence" properties; cells utilize a heterogeneous transition function inspired by linear genetic programming; and there exists a notion of genetic transfer of transition functions between adjacent cells
HetCA exhibit longterm phenotypic dynamics, sustaining a high level of variance
over very long runs and displayed greater behavioral diversity than classical cellular automata\cite{medernach2013long}. HetCA also display Shanahan \cite{shanahan2012evolutionary} definition of evolutionary progress on three criteria: robustness, size and density of generated genotypes\cite{medernach2015evolutionary}.

The eukaryotic chromatin remodeling machinery, the cell cycle regulation
systems \cite{koonin2002origin}, the nuclear envelope, the cytoskeleton, and the
programmed cell death (PCD, or apoptosis) apparatus all are
such major eukaryotic innovations, which do not appear to
have direct prokaryotic predecessors. 

\section{Experimental Setup}
Originally, in HetCA, the new genotype of a cell was randomly chosen among candidates genotypes. To introduce environmental variations, we chose to vary the likelihood of spread of the genotype of a cell according to the state of this cell. The chances of the candidate genotype of the cell $c$ to be selected are then: $c=K(S_c)/\sum_{i=1}^{n} K(S_{c_i})$ with $S_c$ state of the cell $c$, $K(S)$ likelihood of spread of state $S$ and $n$ number of candidate genotypes. Therefore an environment is characterized by the odds of propagation of the five living states $\{K(S_1),K(S_2),K(S_3),K(S_4),K(S_5)\}$.   
To mimic environmental fluctuations we initialise the simulation with $K(S_i)=1  \forall i \in [1,5]$ and then we regularly change those values from iteration 3000 of the cellular automata. 

We chose to introduce four forms of environmental fluctuations described in Table~\ref{tab:environments}.

\noindent The \emph{Short-cycle Fluctuation} : consists of alternating between two environments every 100 iterations of the cellular automaton. We chose to vary the environment every 100 iterations to stay in the same range of frequency as in Lispson \cite{lipson2002origin} examples 20 and 100 generations and \cite{yu2007program} experiments 10, 20 and 50 generations. In fact we consider that a successful reproductive cycle involves passing a cell through the quiescent state. And this should takes between two iterations (alternating between the quiescent state and a living state) and seven iterations  (if a cell remains in a living state four consecutive iterations it goes to the decay state and can not receive genotype for an important period).

\noindent The \emph{Light Fluctuation} : consists of alternating between five environments every 2500 iterations of the cellular automaton. The first fives each prohibit a different state from the five living state, the latter gives equal chance to each of the five states.

\noindent The \emph{Strong Fluctuation} : consists of alternating between twelve environments every 2500 iterations of the cellular automaton. The first eleven each prohibit a different combination of two states from the five living state, the latter gives equal chance to each of the five states.

\noindent The \emph{Gradual Fluctuation} : is similar to the strong fluctuation except that it include a transition phase of $T=60$ iterations in between two environment where spread likehood values progressively switch from the value a previous environment to the ones of the new environment. During this phase the sate spreading is defined by the following formula : $K(S,t)=K_p(S) \times (T-t) + K_{p+1}(S) \times t$ where $t$ is the number of iteration achieved since the beginning of the transition phase; $K_p(S)$ and $K_{p+1}(S)$ are likelihood of spread of state $S$ for the curent environment and the next environment. 

Owing to the variety of biological temporal rhythms and reproductive cycles the relevance of the following analogy is certainly limited . However \emph{short-cycle fluctuation} would be similar to what are the circadian rhythms for some bacteria: very regular cycles in which these organisms have enough time to reproduce several times. While \emph{light fluctuation} would be closer to seasonal fluctuations and \emph{strong and gradual fluctuations} would akin ecological crisis. 


\newcommand*{\TitleParbox}[1]{\parbox[l]{10cm}{\raggedright #1}}%
\begin{table*}
\caption{Environments.\label{tab:environments}}
\scriptsize

\begin{tabular}{lccccl}
\toprule%
{\textbf{Name}} & {\textbf{Short Name}} & \textbf{Cycles}\tnote{a} & \textbf{Transitions}\tnote{b} &\textbf{Environment list} \tabularnewline
\toprule%
\textbf{Stable Environment} & [SE] & NA & NA & \TitleParbox{$\{1,1,1,1,1\}$} \tabularnewline

\textbf{Short-cycle Fluctuations} & [ScF] & 100 & 1 & \TitleParbox{$\{1,1,1,0,0\}$, $\{1,1,1,0,0\}$} \tabularnewline

\textbf{Light Fluctuations} & [LF] & 2500 &  1 & \TitleParbox{$\{1,1,1,1,0\}$, $\{1,1,1,0,1\}$, $\{1,1,0,1,1\}$, $\{1,0,1,1,1\}$, $\{0,1,1,1,1\}$, $\{1,1,1,1,1\}$}\tabularnewline
    
\textbf{Strong Fluctuations} & [SF] & 2500 & 1 & \TitleParbox{$\{0,0,1,1,1\}$, $\{1,1,1,0,0\}$, $\{0,1,0,1,1\}$, $\{1,0,1,1,0\}$, $\{0,1,1,0,1\}$, $\{1,1,0,1,0\}$, $\{1,0,1,0,1\}$, $\{0,1,1,1,0\}$, $\{1,0,0,1,1\}$, $\{1,1,0,0,1\}$, $\{1,1,1,1,1\}$} \tabularnewline

\textbf{Gradual Fluctuations} & [GF] & 2500  & 60 & \TitleParbox{$\{0,0,1,1,1\}$, $\{1,1,1,0,0\}$, $\{0,1,0,1,1\}$, $\{1,0,1,1,0\}$, $\{0,1,1,0,1\}$, $\{1,1,0,1,0\}$, $\{1,0,1,0,1\}$, $\{0,1,1,1,0\}$, $\{1,0,0,1,1\}$, $\{1,1,0,0,1\}$, $\{1,1,1,1,1\}$} \tabularnewline

\bottomrule%
\end{tabular}%
\end{table*} 



\section{Results}

\section{Discussions}


\section{Conclusions}
\bibliography{gp-bibliography,David}{}

\bibliographystyle{plain}
\bibliographystyle{abbrv}
\end{document}
