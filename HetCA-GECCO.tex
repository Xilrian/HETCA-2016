% This is "sig-alternate.tex" V2.1 April 2013
% This file should be compiled with V2.5 of "sig-alternate.cls" May 2012
%
% This example file demonstrates the use of the 'sig-alternate.cls'
% V2.5 LaTeX2e document class file. It is for those submitting
% articles to ACM Conference Proceedings WHO DO NOT WISH TO
% STRICTLY ADHERE TO THE SIGS (PUBS-BOARD-ENDORSED) STYLE.
% The 'sig-alternate.cls' file will produce a similar-looking,
% albeit, 'tighter' paper resulting in, invariably, fewer pages.
%
% ----------------------------------------------------------------------------------------------------------------
% This .tex file (and associated .cls V2.5) produces:
%       1) The Permission Statement
%       2) The Conference (location) Info information
%       3) The Copyright Line with ACM data
%       4) NO page numbers
%
% as against the acm_proc_article-sp.cls file which
% DOES NOT produce 1) thru' 3) above.
%
% Using 'sig-alternate.cls' you have control, however, from within
% the source .tex file, over both the CopyrightYear
% (defaulted to 200X) and the ACM Copyright Data
% (defaulted to X-XXXXX-XX-X/XX/XX).
% e.g.
% \CopyrightYear{2007} will cause 2007 to appear in the copyright line.
% \crdata{0-12345-67-8/90/12} will cause 0-12345-67-8/90/12 to appear in the copyright line.
%
% ---------------------------------------------------------------------------------------------------------------
% This .tex source is an example which *does* use
% the .bib file (from which the .bbl file % is produced).
% REMEMBER HOWEVER: After having produced the .bbl file,
% and prior to final submission, you *NEED* to 'insert'
% your .bbl file into your source .tex file so as to provide
% ONE 'self-contained' source file.
%
% ================= IF YOU HAVE QUESTIONS =======================
% Questions regarding the SIGS styles, SIGS policies and
% procedures, Conferences etc. should be sent to
% Adrienne Griscti (griscti@acm.org)
%
% Technical questions _only_ to
% Gerald Murray (murray@hq.acm.org)
% ===============================================================
%
% For tracking purposes - this is V2.0 - May 2012

\documentclass{sig-alternate-05-2015}


\begin{document}

% Copyright
\setcopyright{acmcopyright}
%\setcopyright{acmlicensed}
%\setcopyright{rightsretained}
%\setcopyright{usgov}
%\setcopyright{usgovmixed}
%\setcopyright{cagov}
%\setcopyright{cagovmixed}


% DOI
\doi{10.475/123_4}

% ISBN
\isbn{123-4567-24-567/08/06}

%Conference
\conferenceinfo{PLDI '13}{June 16--19, 2013, Seattle, WA, USA}

\acmPrice{\$15.00}

%
% --- Author Metadata here ---
\conferenceinfo{WOODSTOCK}{'97 El Paso, Texas USA}
%\CopyrightYear{2007} % Allows default copyright year (20XX) to be over-ridden - IF NEED BE.
%\crdata{0-12345-67-8/90/01}  % Allows default copyright data (0-89791-88-6/97/05) to be over-ridden - IF NEED BE.
% --- End of Author Metadata ---

\title{Heterogeneous Cellular Automata Evolution in a Fluctuating Environment}
%
% You need the command \numberofauthors to handle the 'placement
% and alignment' of the authors beneath the title.
%
% For aesthetic reasons, we recommend 'three authors at a time'
% i.e. three 'name/affiliation blocks' be placed beneath the title.
%
% NOTE: You are NOT restricted in how many 'rows' of
% "name/affiliations" may appear. We just ask that you restrict
% the number of 'columns' to three.
%
% Because of the available 'opening page real-estate'
% we ask you to refrain from putting more than six authors
% (two rows with three columns) beneath the article title.
% More than six makes the first-page appear very cluttered indeed.
%
% Use the \alignauthor commands to handle the names
% and affiliations for an 'aesthetic maximum' of six authors.
% Add names, affiliations, addresses for
% the seventh etc. author(s) as the argument for the
% \additionalauthors command.
% These 'additional authors' will be output/set for you
% without further effort on your part as the last section in
% the body of your article BEFORE References or any Appendices.

\numberofauthors{5} %  in this sample file, there are a *total*
% of EIGHT authors. SIX appear on the 'first-page' (for formatting
% reasons) and the remaining two appear in the \additionalauthors section.
%
\author{
% You can go ahead and credit any number of authors here,
% e.g. one 'row of three' or two rows (consisting of one row of three
% and a second row of one, two or three).
%
% The command \alignauthor (no curly braces needed) should
% precede each author name, affiliation/snail-mail address and
% e-mail address. Additionally, tag each line of
% affiliation/address with \affaddr, and tag the
% e-mail address with \email.
%
% 1st. author
\alignauthor
XXX\\
       \affaddr{YYY}\\
       \affaddr{YYY}\\
       \affaddr{YYY}\\
       \email{Y@Y.Y}
% 2nd. author
\alignauthor
XXX\\
       \affaddr{YYY}\\
       \affaddr{YYY}\\
       \affaddr{YYY}\\
       \email{Y@Y.Y}
% 3rd. author
\alignauthor 
XXX\\
       \affaddr{YYY}\\
       \affaddr{YYY}\\
       \affaddr{YYY}\\
       \email{Y@Y.Y}
\and  % use '\and' if you need 'another row' of author names
% 4th. author
\alignauthor 
XXX\\
       \affaddr{YYY}\\
       \affaddr{YYY}\\
       \affaddr{YYY}\\
       \email{Y@Y.Y}
% 5th. author
\alignauthor 
XXX\\
       \affaddr{YYY}\\
       \affaddr{YYY}\\
       \affaddr{YYY}\\
       \email{Y@Y.Y}
}
% There's nothing stopping you putting the seventh, eighth, etc.
% author on the opening page (as the 'third row') but we ask,
% for aesthetic reasons that you place these 'additional authors'
% in the \additional authors block, viz.
% Just remember to make sure that the TOTAL number of authors
% is the number that will appear on the first page PLUS the
% number that will appear in the \additionalauthors section.

\maketitle
\begin{abstract}
The importance of environnemental fluctuations in the evolution of living beings by natural selection have been widely noted by biologists and have been linked to many important characteristics of the life as: modularity, plasticity, size of the genotype, mutation rate, learning or epigenetic adaptations. However, in artificial life simulations, environmental fluctuations are usually seen as a problem to be solved rather than an essential characteristic of evolution. We propose in this paper to use HetCA, an heterogeneous cellular automata characterized by its ability to generate open ended long-term evolution and evolutionary progress, to measure the impact of different forms of environmental fluctuations. Our results indicate that environnemental fluctuations induce mechanisms analogous to epigenetic adaptations in HetCA.
\end{abstract}


%
% The code below should be generated by the tool at
% http://dl.acm.org/ccs.cfm
% Please copy and paste the code instead of the example below. 
%
\begin{CCSXML}
<ccs2012>
 <concept>
  <concept_id>10010520.10010553.10010562</concept_id>
  <concept_desc>Computer systems organization~Embedded systems</concept_desc>
  <concept_significance>500</concept_significance>
 </concept>
 <concept>
  <concept_id>10010520.10010575.10010755</concept_id>
  <concept_desc>Computer systems organization~Redundancy</concept_desc>
  <concept_significance>300</concept_significance>
 </concept>
 <concept>
  <concept_id>10010520.10010553.10010554</concept_id>
  <concept_desc>Computer systems organization~Robotics</concept_desc>
  <concept_significance>100</concept_significance>
 </concept>
 <concept>
  <concept_id>10003033.10003083.10003095</concept_id>
  <concept_desc>Networks~Network reliability</concept_desc>
  <concept_significance>100</concept_significance>
 </concept>
</ccs2012>  
\end{CCSXML}

\ccsdesc[500]{Computer systems organization~Embedded systems}
\ccsdesc[300]{Computer systems organization~Redundancy}
\ccsdesc{Computer systems organization~Robotics}
\ccsdesc[100]{Networks~Network reliability}


%
% End generated code
%

%
%  Use this command to print the description
%
\printccsdesc

% We no longer use \terms command
%\terms{Theory}

\keywords{Cellular automata; environnemental fluctuations; Open-Ended Evolution; Epigenetic; Evolvability}

\section{Introduction}
If early population genetics theory assumed the environment to be constant, since Richards Levins works \cite{levins1968evolution} in the 70 up to Evolution in Four Dimensions by Eva Jablonka and al. \cite{jablonka2014evolution} the importance of environnemental fluctuations in the evolution of living beings by natural selection have been widely noted. Through those works and others, environmental fluctuations have been linked to many important questions about the mechanisms of evolution such as: modularity, plasticity, size of the genotype, mutation rate and through this the evolvability. Recently Jablonka\cite{jablonka2014evolution} have developed the idea that epigenetic mechanisms controlling the rate of changes in certain genes have evolved to counter frequent environmental changes. 
Environmental changes can include introduction of new predator or new potencial food, radical environment modifications such as climate change inducing environmental stress but also cyclic changes such as daily cycle of light and darkness, seasons...  

In \cite{lachmann1996inheritance} Lachemann model the consequences of such cycling variations on phenotypical inheritances. Their model predict that, when those cycles are longer than the reproductive cycle but relatively short heritable variations produced by non-DNA inheritance systems are likely to be observed.

Many work in artificial life approach the issue of environmental variations, but the greater part are interested mainly in the spacial variations, or considering them as a problem to solve.

This paper is organized as follows. Section 2 highlights consequences of variable environment in biological evolution. In Section 3, we explain the mecanisms of HetCA simulation. The implementation of environmental fluctuation in HetCA  is then explained in Section 4. Section 5 details the computer experimental setup while we reports experimental results in Section 6. We discuss the implications of these results in Section 7. Finally, Section 8 concludes the paper.


\section{Background}
In \cite{lipson2002origin} Lipson demonstrated a correlation between the modularity and the rate of change of the environment resources. While in \cite{yu2007program} Yu used  observed populations exploit neutrality to cope with environmental fluctuations and therefore evolve some sort of evolvability under 2 alternating objective functions. Both simulations used Genetic Programming (GP) and fitness explicit functions.

\section{HetCA}

The eukaryotic chromatin remodeling machinery, the cell cycle regulation
systems \cite{koonin2002origin}, the nuclear envelope, the cytoskeleton, and the
programmed cell death (PCD, or apoptosis) apparatus all are
such major eukaryotic innovations, which do not appear to
have direct prokaryotic predecessors. 

\section{Experimental Setup}

\section{Results}

\section{Discussions}


\section{Conclusions}
\bibliography{gp-bibliography,David}{}

\bibliographystyle{plain}
\bibliographystyle{abbrv}
\end{document}
