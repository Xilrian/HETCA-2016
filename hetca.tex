HetCA \citep{medernach2013long} is based on classical two-dimensional CA with several additional features: cells follow a heterogeneous transition function, i.e.~one which depends on their location, inspired by linear genetic programming~(LGP); they can also fall into special \emph{decay} and \emph{quiescent} states; and there is a notion of genetic transfer of transition functions (i.e.~genotypes) between adjacent cells. Decay and quiescent cells do not possess a genotype; all other cells do, and are called \emph{living}; there are 5 different living states; quiescent cells can acquire a genotype from any nearby living cell and therefore become living in turn; decay cells cannot, but become quiescent after a number of consecutive iterations comprised between 375 and 1,875 after decay. Living cells always automatically turn into decay after 7 consecutive iterations spent in any one or several living states (tracked by an ``age'' counter).

We showed that HetCA could exhibit long-term phenotypic dynamics \citep{medernach2013long}, a high level of variance over very long runs, greater behavioral diversity than classical CA, and ``evolutionary progress'' \cite{shanahan2012evolutionary} on three criteria: robustness, size and density of the genotype \citep{medernach2015evolutionary}.
%Moreover, some emergent properties of HetCA are similar to two of the five major eukaryotic innovations which do not appear to have direct prokaryotic predecessors defined in \citep{smith1997major} as : (1)~the eukaryotic chromatin remodelling machinery; (2)~the cell cycle regulation systems; (3)~the nuclear envelope; (4)~the cytoskeleton; and (5)~the apoptosis apparatus \citep{koonin2002origin}. Indeed, in HetCA, the loss of the genotype of the cell as it turns to the quiescent state may seem similar to apoptosis while survival strategies such as the ones depicted in Figure~\ref{foursteps} might akin cell cycle regulation systems \todo{Un peu "Gros" pour la taille des explications non? je veux dire que si c'est vraiment le cas, il faudrait détailler un peu?}

Finally, while there is a lasting debate over the units of selection in evolutionary biology since the origins of the field: genotype selection, phenotype selection, epigenetic selection, behavioral selection, multilevel selection, group selection, and so on \citep{lloyd2012unitsandlevelsofselection,okasha2006evolution}, several of them are potentially included in HetCA. There is genotypic selection of the transition rules, but also phenotypic selection of cell groups able to replicate patterns such as the ones found in the Game of Life. This point is important when one is interested in environmental fluctuations because, as mentioned in the introduction, we anticipate that the existence of frequent environmental fluctuations will promote phenotypic selection over genotypic selection.
