HetCA is based on classical two-dimensional cellular automata~(CA), but has several added features: cells have \emph{age}, \emph{decay} and \emph{quiescence} properties; cells utilize a heterogeneous transition function inspired by linear genetic programming~(LGP); and there exists a notion of genetic transfer of transition functions between adjacent cells.

HetCA exhibits long term phenotypic dynamics, sustains a high level of variance over very long runs and displays greater behavioural diversity than classical cellular automata\citep{medernach2013long}. HetCA also exhibits Evolutionary Progress (EP) according to the three criteria : robustness, size and density of generated genotypes\citep{medernach2015evolutionary}, using Shanahan's definition of EP \citep{shanahan2012evolutionary}.

Moreover, some emergent properties of HECA are similar to two of the five major eukaryotic innovations which do not appear to have direct prokaryotic predecessors defined in~\citep{smith1997major} as : the eukaryotic chromatin remodelling machinery; the cell cycle regulation systems; the nuclear envelope, the cytoskeleton; and the apoptosis apparatus\citep{koonin2002origin}. Indeed, in HetCA, the loss of the genotype of the cell as it turns to the quiescent state may seem similar to apoptosis while survival strategies such as the ones depicted in Figure~\ref{foursteps} might akin cell cycle regulation systems.
Finally, controversy over the units of selection\footnote{Genotype selection, Phenotype selection, epigenetic selection, comportemental selection, multilevel selection...} in evolutionary biology~\citep{okasha2006evolution} are numerous and date back to the origins of this field of research and there is potentially several units of selection in HetCA: genotypic selection of the transition rules but also phenotypic selection of cell group replicating patterns that one can also find in cellular automata such as game of life.  This issue is important when one is interested in environmental fluctuations because, as mentioned in introduction, it is anticipated that the existence of frequent environmental fluctuation could promote phenotypic selection compared to genotypic selection.


