HetCA is based on classical two-dimensional Cellular Automata~(CA), but has several additional features: cells utilize a heterogeneous transition function inspired by linear genetic programming~(LGP); a notion of genetic transfer of transition functions (genotypes) between adjacent cells has been implemented; and cells use \emph{decay} and \emph{quiescence} special states. Normal, non-\emph{decay} and non-\emph{quiescent} cells are called \emph{living cells}. \emph{Decay} cells and \emph{quiescent} cells don't have any genotype, \emph{quiescent} cells can acquire one from nearby living cells and therefore become a living cell, \emph{decay} can't but will turn to quiescent state after between 375 and 1875 consecutive iterations from the iteration it turn to decay. Living cells will turn automatically to decay after remaining 7 consecutive iterations at a living state. 

HetCA has been shown to exhibit long term phenotypic dynamics, high level of variance over very long run, a greater behavioral diversity than classical CA and evolves progress other three criteria (robustness size and density) given the~\cite{shanahan2012evolutionary}'s definition of Evolutionary Progress~\citep{medernach2015evolutionary}.

%Moreover, some emergent properties of HetCA are similar to two of the five major eukaryotic innovations which do not appear to have direct prokaryotic predecessors defined in~\citep{smith1997major} as : (1)~the eukaryotic chromatin remodelling machinery; (2)~the cell cycle regulation systems; (3)~the nuclear envelope; (4)~the cytoskeleton; and (5)~the apoptosis apparatus~\citep{koonin2002origin}. Indeed, in HetCA, the loss of the genotype of the cell as it turns to the quiescent state may seem similar to apoptosis while survival strategies such as the ones depicted in Figure~\ref{foursteps} might akin cell cycle regulation systems \todo{Un peu "Gros" pour la taille des explications non? je veux dire que si c'est vraiment le cas, il faudrait détailler un peu?}

Finally, controversy over the units of selection\footnote{Genotype selection, Phenotype selection, epigenetic selection, comportemental selection, multilevel selection \cite{lloyd2012unitsandlevelsofselection}...} in evolutionary biology~\citep{okasha2006evolution} date back to the origins of this field up to now and there is potentially several units of selection in HetCA: genotypic selection of the transition rules but also phenotypic selection of cell group replicating patterns that one can also find in cellular automata such as game of life.  This issue is important when one is interested in environmental fluctuations because, as mentioned in introduction, it is anticipated that the existence of frequent environmental fluctuations could promote phenotypic selection compared to genotypic selection.


