Environmental changes may include cyclic events such as seasonal changes or a daily cycle of light and darkness, occasional changes such as the introduction of new predator or the potential for a new source of food, or much more radical modifications such as environmental stress induced by climate change.

Early population genetics theory assumed the environment to be constant\todo{citation? Fisher? }. However, since then, work such as \citep{levins1968evolution} or more recently \citep{jablonka2014evolution}, have put the emphasis on the importance of environmental fluctuations in the evolution by natural selection of living beings\todo{Not sure about this sentence.. ``With the integration of the epigenitec (EIS), developmetnmal (EVODEVO) and environmental (NICHE CONTRUCTION \cite{laland2016anintroductiontonicheconstructiontheory}) dimension to the study of evolutionary biology ''??}. Through these and other works many central questions about the mechanisms of evolution such as modularity, plasticity~\citep{west2005developmental}, size of the genotype, mutation rate and evolvability have been linked to environmental changes. \cite{jablonka2014evolution} stress that a changeable environment will unmask variations in the capacity of individuals to make adjustments to changed conditions and therefore promote plasticity. They advance that \say{For a lineage in a constantly changing environment, switching among several alternative heritable states was probably an advantage. While cells in one state survived in one set of conditions, those in other states did better in different circumstances.}\emph{(ibid. p. 318)}. For them, constantly changing environment or cycling variations might explain the origin of the Epigenetic Inheritance Systems \todo{add citation (other than jablonka maybe \cite{heard2014transgenerationalepigeneticinheritancemythsandmechanisms}?)} considering that epigenetic mutations are more reversibles and occure more frequently than genetic mutations. 

\cite{lachmann1996inheritance} have modelled the effects of cycling variations, such as seasonal or daily cycles, on phenotypical inheritances. Their model predicted that when the investigated cycles are longer than the reproductive cycle but relatively short otherwise, heritable variations produced by non-DNA inheritance systems are likely to be observed. \todo{Ça semble un peu parachuter d enul part non? pas de lien avec ce qu'il y a avant ou après}

Moreover, a proportion of existing work in artificial life, especially in evolutionary robotics \citep{floreano2000evolutionary}, considers the issue of environmental variations and among that literature some have explicitly define the environment as a driving evolutionary power \citep{bredeche2012environmentdrivenopenende} however as far as we know, nobody has systematically studied the general properties induced by such environmental fluctuation and a significant majority of the other works in that field still consider environmental changes only as a problem to be solved.   %However, it is probably fair to say that, a significant majority of these previous work has either focused on aspects such as normal spatial variation, or has viewed environmental change as a problem to be solved.

In this paper we propose a way to fill that gap and we propose an experimental setup that allows to mesure the influence of environmental fluctuations on the course of the evolution of a cellular automata. We show that such fluctuation lead to the emergence of strategies similar to those encontr

This paper is organized as follows. Section~\ref{sec:bground}, we explain the mechanisms of HetCA simulation. The implementation of environmental fluctuation in HetCA is then explained in Section~\ref{sec:exsetup}. Section 5 details the computer experimental setup while we reports experimental results in Section~\ref{sec:results}. We discuss the implications of these results in Section~\ref{sec:discuss}. Finally, Section~\ref{sec:conc} concludes the paper.



