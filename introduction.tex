Environmental changes may include cyclic events such as seasonal changes or a daily cycle of light and darkness, occasional changes such as the introduction of new predator or the potential for a new source of food, or much more radical modifications such as environmental stresses induced by climate change.

Early population genetics theory assumed the environment to be constant\todo{citation? Fisher? }. However, since then, work such as~\citep{levins1968evolution} or more recently~\citep{jablonka2014evolution}, as well as the integration of the epigenetic (EIS), developmental (the "Evo-Devo" school \cite{muller2007evodevoextendingtheevolutionarysynthesis) and environmental (as done by poeple studying "Niche Construction" mechanisms~\cite{laland2016anintroductiontonicheconstructiontheory} dimensions to evolutionary biology, have put the emphasis on the importance of environmental fluctuations in the evolution by natural selection of living beings. Through these and other works many central questions about the mechanisms of evolution such as modularity, plasticity~\citep{west2005developmental}, size of the genotype, mutation rate and evolvability have been linked to environmental changes. \cite{jablonka2014evolution} stress that a changeable environments will unmask variations in the capacity of individuals to make adjustments to changed conditions and therefore promote plasticity. They advance that \say{For a lineage in a constantly changing environment, switching among several alternative heritable states was probably an advantage. While cells in one state survived in one set of conditions, those in other states did better in different circumstances.}\emph{(ibid. p. 318)}. For them, constantly changing environment or cycling variations might explain the origin of the Epigenetic Inheritance Systems~\cite{heard2014transgenerationalepigeneticinheritancemythsandmechanisms} as epigenetic mutations are more reversibles and occur more frequently than genetic mutations. 

Among the work done on the topic, \cite{lachmann1996inheritance} have modelled the effects of cycling variations, such as seasonal or daily cycles, on phenotypical inheritances. Their model predicted that when the investigated cycles are longer than the reproductive cycle but relatively short otherwise, heritable variations produced by non-DNA inheritance systems are likely to be observed. 

On the other hand a proportion of existing works in artificial life, especially in evolutionary robotics~\citep{floreano2000evolutionary}, considers the issue of environmental variations. Some in this literature have explicitly define the environment as a driving evolutionary force~\citep{bredeche2012environmentdrivenopenende}. However and as far as we know, a significant majority of the works in that field still consider environmental changes only as a problem to be solved and nobody has systematically studied the general properties induced by environmental fluctuation over evolutionary dynamics.  %However, it is probably fair to say that, a significant majority of these previous work has either focused on aspects such as normal spatial variation, or has viewed environmental change as a problem to be solved.

Nonetheless~\cite{lipson2002origin} demonstrated a correlation between the modularity and the rate of change of the environment resources. While in \cite{yu2007program} observed that populations exploit neutrality to cope with environmental fluctuations and therefore evolve some sort of evolvability under two alternating objective functions. Both of these simulations used Genetic Programming~(GP) and employed explicit fitness functions.

Taking some of the observations made in those two last studies, we propose an experimental setup that allows to quantitatively measure the influence of cyclic environmental fluctuations on the course of the evolution of a Cellular Automata (CA). We show that such fluctuations lead to the emergence of processes similar to those exhibited by epigenetic inheritance systems. \todo{Revoir la transition entre ces deux paragraphes: que reprends-t-on des deux études? que change-t-on? quelles sont les hypothese?}

The paper is organized as follow. Section~\ref{sec:bground}, explains the general mechanisms of HetCA simulation and the implementation of environmental fluctuations in HetCA is then detailed in Section~\ref{sec:exsetup}. Section\ref{sec:method} describes the computer experimental setup while we report experimental results in Section~\ref{sec:results}. We discuss the implications of these results in Section~\ref{sec:discuss}. Finally, Section~\ref{sec:conc} concludes the paper.

