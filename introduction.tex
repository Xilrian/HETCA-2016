
Environmental changes may include cyclic events such as seasonal changes or a daily cycle of light and darkness, occasional changes such as the introduction of new predator or the potential for a new source of food, or much more radical modifications such as environmental stresses induced by climate change.

Since works such as~\citep{levins1968evolution} or more recently~\citep{jablonka2014evolution}, as well as with the attempt made to integrate the epigenetic (EIS), developmental ("Evo-Devo"~\citep{muller2007evodevoextendingtheevolutionarysynthesis}) and environmental ("Niche Construction"~\citep{laland2016anintroductiontonicheconstructiontheory} dimensions to evolutionary biology, the emphasis has been put on the importance of such change on the course of the evolution by natural selection of living beings. Through these and other works, they have been linked to many central mechanisms of evolution. Modularity, plasticity~\citep{west2005developmental}, size of the genotype, mutation rate and evolvability are some of the most discussed example. 

For \cite{jablonka2014evolution}, changeable environments will unmask variations in the capacity of individuals to make adjustments to changed conditions and therefore promote plasticity. They advance that \say{For a lineage in a constantly changing environment, switching among several alternative heritable states was probably an advantage. While cells in one state survived in one set of conditions, those in other states did better in different circumstances.}\emph{(ibid. p. 318)}. Following that line of thought, constantly changing environment or cycling variations might explain the origin of the Epigenetic Inheritance Systems~\citep{heard2014transgenerationalepigeneticinheritancemythsandmechanisms}, as epigenetic mutations are more reversibles and occur more frequently than genetic mutations. 

To illustrate that, \cite{lachmann1996inheritance} have modelled the effects of cycling variations, such as seasonal or daily cycles, on phenotypical inheritances. Their model predicted that indeed when the investigated cycles are longer than the reproductive cycle but relatively short otherwise, heritable variations produced by non-DNA inheritance systems are likely to be observed. 

In parallel to such studies, a proportion of existing works in artificial life, especially in evolutionary robotics~\citep{floreano2000evolutionary}, considers the issue of environmental variations. Some work in this literature have explicitly define the environment as a driving evolutionary force~\citep{bredeche2012environmentdrivenopenende}. %However and as far as we know, a significant majority of the works in that field still consider environmental changes only as a problem to be solved and nobody has systematically studied the general properties induced by environmental fluctuation over evolutionary dynamics.  %However, it is probably fair to say that, a significant majority of these previous work has either focused on aspects such as normal spatial variation, or has viewed environmental change as a problem to be solved.

Other, such as~\cite{lipson2002origin}, demonstrated a correlation between the modularity and the rate of change of the environment resources, while \cite{yu2007program} observed that populations exploit neutrality to cope with environmental fluctuations and therefore evolve some sort of evolvability under two alternating objective functions. Both of these two last simulations used Genetic Programming~(GP) and employed explicit fitness functions.

We wish to study effects of such fluctuations in a model closer to the living world without using any explicit fitness function. We then propose an open ended experimental setup that allows to systematically and quantitatively measure the influence of cyclic environmental fluctuations on the course of the evolution of a Cellular Automata (CA). We show that such fluctuations lead to the emergence of processes similar to those exhibited by epigenetic inheritance systems.

The paper is organized as follow. First the general mechanisms of HetCA model are described. Then implementation of environmental fluctuations in HetCA is detailed. Afterward we specify the computer experimental setup used to study environmental fluctuations. Later we report experimental results and discuss the implications of these results. Finally we propose a qualitative analysis and conclude.

