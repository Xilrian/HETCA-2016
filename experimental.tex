
\subsection{Environmental Fluctuations}
 As in the previous version of HetCA the genotype of an individual is its transition rules encoded with CA-LGP using the function set depicted in Table~\ref{funcSet}. The mutation of genotypes is enabled and we use the Micro/Marco-mutation of CA-LGP as described in \citep{medernach2013long}.
 
 
 \begin{table}
\scriptsize
\centering
  \begin{tabular}{l>{\centering}p{0.6\columnwidth}}
  \toprule%
    \textbf{op. name}	& \textbf{action} on inputs $(x,y)$\tabularnewline
 \toprule%   
    abs			& $|x|$ \tabularnewline
    plus		& $x+y$ \tabularnewline
    delta		& 1, if $|x-y| < 1/10000$; 0 o.w. \tabularnewline
    dist		& $|x-y|$ \tabularnewline
    inv			& $1-x$ \tabularnewline
    inv2		& safeDiv($1, x$) \tabularnewline
    magPlus		& $|x+y|$ \tabularnewline
    max			& $\max \{x,y\}$ \tabularnewline
    min			& $\min \{x,y\}$ \tabularnewline
    safeDiv		& $x/y$ if $|y| >  1/10000$; 1 o.w. \tabularnewline
    safePow		& $x^y$, if defined; 1 o.w. \tabularnewline
    thresh		& 1, if $x > y$; 0 o.w.\tabularnewline
    times		& $xy$ \tabularnewline
    zero		& 1, if $|x| < 1/10000$; 0 o.w. \tabularnewline
\bottomrule%
  \end{tabular}
    \caption{\textbf{Function set}. \label{funcSet}}
\end{table}

But in order to introduce environmental variations, where the new genotype of a cell was originally randomly chosen among candidate genotypes we vary the likelihood of spread of the genotype of a cell according to the state of this cell. In this new setup, the chances $P(c)$ of the candidate genotype of the cell $c$ to be selected are then: $P(c)=K(S_c)/\sum_{i=1}^{n} K(S_{c_i})$ with $S_c$ state of the cell $c$, $K(S)$ likelihood of spread of state $S$ and $n$ number of candidate genotypes. Therefore an environment $E$ is characterized by the odds of propagation of the five living states $E=\{K(S_1),K(S_2),K(S_3),K(S_4),K(S_5)\}$.   
To mimic environmental fluctuations we initialise the simulation with $K(S_i)=1  \forall i \in [1,5]$ and then we regularly change those values every $f$ iterations from iteration 3000 of the cellular automata. 

We chose to introduce four forms of environmental fluctuations described in Table~\ref{tab:environments}.

\newcommand*{\TitleParbox}[1]{\parbox[l]{10cm}{\raggedright #1}}%
\begin{table*}
\caption{Environments.\label{tab:environments}}
\scriptsize

\begin{tabular}{lccccl}
\toprule%
{\textbf{Name}} & {\textbf{Short Name}} & \textbf{Cycles}\tnote{a} & \textbf{Transitions}\tnote{b} &\textbf{Environment list} \tabularnewline
\toprule%
\textbf{Stable Environment} & [SE] & NA & NA & \TitleParbox{$\{1,1,1,1,1\}$} \tabularnewline

\textbf{Short-cycle Fluctuations} & [ScF] & 100 & 1 & \TitleParbox{$\{1,1,1,0,0\}$, $\{1,1,1,0,0\}$} \tabularnewline

\textbf{Light Fluctuations} & [LF] & 5000 &  1 & \TitleParbox{$\{1,1,1,1,0\}$, $\{1,1,1,0,1\}$, $\{1,1,0,1,1\}$, $\{1,0,1,1,1\}$, $\{0,1,1,1,1\}$, $\{1,1,1,1,1\}$}\tabularnewline
    
\textbf{Strong Fluctuations} & [SF] & 5000 & 1 & \TitleParbox{$\{0,0,1,1,1\}$, $\{1,1,1,0,0\}$, $\{0,1,0,1,1\}$, $\{1,0,1,1,0\}$, $\{0,1,1,0,1\}$, $\{1,1,0,1,0\}$, $\{1,0,1,0,1\}$, $\{0,1,1,1,0\}$, $\{1,0,0,1,1\}$, $\{1,1,0,0,1\}$, $\{1,1,1,1,1\}$} \tabularnewline

%\textbf{Gradual Fluctuations} & [GF] & 5000  & 60 & \TitleParbox{$\{0,0,1,1,1\}$, $\{1,1,1,0,0\}$, $\{0,1,0,1,1\}$, $\{1,0,1,1,0\}$, $\{0,1,1,0,1\}$, $\{1,1,0,1,0\}$, $\{1,0,1,0,1\}$, $\{0,1,1,1,0\}$, $\{1,0,0,1,1\}$, $\{1,1,0,0,1\}$, $\{1,1,1,1,1\}$} \tabularnewline

\bottomrule%
\end{tabular}%
\end{table*} 

\noindent \emph{Short-cycle Fluctuation}: consists of alternating between two environments every 100 iterations of the cellular automaton. We chose to vary the environment every $f=100$ iterations to stay in the same range of frequency as described in Lispson~\citep{lipson2002origin} examples 20 and 100 generations and~\citep{yu2007program} experiments 10, 20 and 50 generations. In fact we consider that a successful reproductive cycle involves passing a cell through the quiescent state. And this should take between two iterations (alternating between the quiescent state and a living state) and seven iterations (if a cell remains in a living state more than seven consecutive iterations it goes to the decay state and can not receive a genotype for an important period).

\noindent \emph{Light Fluctuation}: consists of alternating between five environments every $f=5000$ iterations of the cellular automaton. The first five each prohibit a different state from the five living state, the latter gives equal chance to each of the five states.

\noindent \emph{Strong Fluctuation}: consists of alternating between twelve environments every $f=5000$ iterations of the cellular automaton. The first eleven each prohibit a different combination of two states from the five living states, the latter gives equal chance to each of the five states.

%\noindent \emph{Gradual Fluctuation}: is similar to strong fluctuation except that it includes a transition phase of $T=60$ iterations in between two environments where likelihood of spread of state values progressively switch from the value a previous environment to the ones of the new environment. Over this phase the state spreading is defined by the following formula : $K(S,t)=K_p(S) \times (T-t) + K_{p+1}(S) \times t$ where $t$ is the number of iterations completed since the beginning of the transition phase; $K_p(S)$ and $K_{p+1}(S)$ are likelihood of state spread $S$ for the current environment and the next environment respectively.

\emph{Short-cycle fluctuation} may be analogous to the circadian rhythms for some bacteria: very regular cycles in which these organisms have enough time to reproduce several times. While \emph{light fluctuation} may be similar to seasonal fluctuations and \emph{strong fluctuations} would akin ecological crisis. Although, owing to the variety of both biological temporal rhythms and reproductive cycles, the relevance of these analogies may be limited. 

%\subsection{Common Settings}\label{sec:commonset}


