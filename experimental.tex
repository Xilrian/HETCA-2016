As in the previous version of HetCA \citep{medernach2013long}, the genotype of an individual consists of its transition rules encoded in a custom CA-LGP program using the function set listed in Table~\ref{funcSet}. Such a program maps the space of neighborhood states to a new cell state, while providing an evolvable representation framework based on an alphabet of elementary functions. Individual genotypes are modified by micro-mutations (change in one component of a statement) and macro-mutations (addition or removal of an entire statement) of the corresponding CA-LGP programs.
 
\begin{table}
\caption{Function set.\label{funcSet}}
\scriptsize
\centering
\begin{tabular}{l>{\centering}p{0.6\columnwidth}}
\toprule%
\textbf{operator}	& \textbf{action} on inputs $(x,y)$\tabularnewline
 \toprule%   
    abs			& $|x|$ \tabularnewline
    plus		& $x+y$ \tabularnewline
    delta		& 1, if $|x-y| < 1/10,000$; 0 o.w. \tabularnewline
    dist		& $|x-y|$ \tabularnewline
    inv			& $1-x$ \tabularnewline
    inv2		& safeDiv($1, x$) \tabularnewline
    magPlus		& $|x+y|$ \tabularnewline
    max			& $\max \{x,y\}$ \tabularnewline
    min			& $\min \{x,y\}$ \tabularnewline
    safeDiv		& $x/y$ if $|y| >  1/10,000$; 1 o.w. \tabularnewline
    safePow		& $x^y$, if defined; 1 o.w. \tabularnewline
    thresh		& 1, if $x > y$; 0 o.w.\tabularnewline
    times		& $xy$ \tabularnewline
    zero		& 1, if $|x| < 1/10,000$; 0 o.w. \tabularnewline
\bottomrule%
\end{tabular}
\end{table}

Previously, the new genotype of a living cell $c$ during genetic transfer was chosen randomly among candidate neighboring genotypes according to a uniform distribution. In the present study, to introduce environmental variations we vary the likelihood of propagation of a genotype according to its cell state $S_c$. In this new setup, the probability $P(c)$ of a candidate genotype to be selected becomes $P(c)=K(S_c)/\sum_{i=1}^{n} K(S_{c_i})$, where $K(S)$ the state distribution and $n$ the number of candidate genotypes. Therefore an environment $E$ is characterized by the propagation probabilities of the 5 living states: $E=\{K(S_1),...,K(S_5)\}$. To mimic environmental fluctuations we initialize the simulation with $K(S_i)=1$ everywhere, then regularly modify those values every $f$ iterations starting from iteration 3,000 of the CA. We introduce three types of environmental fluctuations (Table~\ref{tab:environments}):

\newcommand*{\TitleParbox}[1]{\parbox[l]{10cm}{\raggedright #1}}%
\begin{table*}
\caption{Stable and fluctuating environments.\label{tab:environments}}
\scriptsize

\begin{tabular}{lccccl}
\toprule%
{\textbf{Name}} & {\textbf{Short Name}} & \textbf{Cycles}\tnote{a} & \textbf{Transitions}\tnote{b} &\textbf{Environment list: propagation probabilities $E = \{K(S_1),...,K(S_5)\}$ of the living types} \tabularnewline
\toprule%
Stable Environment & (SE) & NA & NA & \TitleParbox{$\{1,1,1,1,1\}$} \tabularnewline

Short-cycle Fluctuations & (ScF) & 100 & 1 & \TitleParbox{$\{0,0,1,1,1\}$, $\{1,1,1,0,0\}$} \tabularnewline

Light Fluctuations & (LF) & 5,000 &  1 & \TitleParbox{$\{1,1,1,1,0\}$, $\{1,1,1,0,1\}$, $\{1,1,0,1,1\}$, $\{1,0,1,1,1\}$, $\{0,1,1,1,1\}$, $\{1,1,1,1,1\}$}\tabularnewline
    
Strong Fluctuations & (SF) & 5,000 & 1 & \TitleParbox{$\{0,0,1,1,1\}$, $\{1,1,1,0,0\}$, $\{0,1,0,1,1\}$, $\{1,0,1,1,0\}$, $\{0,1,1,0,1\}$, $\{1,1,0,1,0\}$, $\{1,0,1,0,1\}$, $\{0,1,1,1,0\}$, $\{1,0,0,1,1\}$, $\{1,1,0,0,1\}$, $\{1,1,1,1,1\}$} \tabularnewline

%\textbf{Gradual Fluctuations} & (GF) & 5,000  & 60 & \TitleParbox{$\{0,0,1,1,1\}$, $\{1,1,1,0,0\}$, $\{0,1,0,1,1\}$, $\{1,0,1,1,0\}$, $\{0,1,1,0,1\}$, $\{1,1,0,1,0\}$, $\{1,0,1,0,1\}$, $\{0,1,1,1,0\}$, $\{1,0,0,1,1\}$, $\{1,1,0,0,1\}$, $\{1,1,1,1,1\}$} \tabularnewline

\bottomrule%
\end{tabular}%
\end{table*} 

\emph{Short-cycle fluctuations (ScF)} consist of alternating between two rather different environments, $\{0,0,1,1,1\}$ and $\{1,1,1,0,0\}$, every $f$ iterations of the CA. We choose $f=100$ to remain within the range of frequency described by \citet{lipson2002origin} and \citet{yu2007program}. Here we consider that a successful reproductive cycle for a cell involves passing through the quiescent state. This should take between 2 iterations (alternating between quiescent and living) and 7 iterations (after which a living cell decays and can no longer receive a genotype for a long period of time).

\emph{Light fluctuations (LF)} consist of alternating between 6 environments every $f=5,000$ iterations. The first 5 environments each prohibit a different living state (out 5 possible ones) from spreading its genotype; the last one gives an equal chance to all living states.

\emph{Strong fluctuations (SF)} consist of alternating between 11 environments every $f=5,000$ iterations. The first 10 environments each prohibit a different pair of living states (out 10 possible ones) from spreading their genotypes; the last one gives an equal chance to all pairs.

%\noindent \emph{Gradual Fluctuation}: is similar to strong fluctuations except that it includes a transition phase of $T=60$ iterations in between two environments where likelihood of spread of state values progressively switch from the value a previous environment to the ones of the new environment. Over this phase the state spreading is defined by the following formula : $K(S,t)=K_p(S) \times (T-t) + K_{p+1}(S) \times t$ where $t$ is the number of iterations completed since the beginning of the transition phase; $K_p(S)$ and $K_{p+1}(S)$ are likelihood of state spread $S$ for the current environment and the next environment respectively.

The rationale behind ScF is their analogy with circadian rhythms in certain bacteria. The intention is to mimic the highly regular cycles during which these organisms have enough time to reproduce repeatedly. LF, by contrast, are more similar to seasonal fluctuations, while SF resemble ecological crises. However, owing to the variety of both biological temporal rhythms and reproductive cycles, the relevance of these analogies remains rather limited.

%\subsection{Common Settings}\label{sec:commonset}
