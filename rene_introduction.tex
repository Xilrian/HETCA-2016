In natural evolution, environmental changes may include cyclic events such as seasonal changes or the daily alternation of light and darkness, occasional changes such as the appearance of new predators or the potential for new food sources, or more radical modifications such as environmental stresses induced by climate transitions.

Since the work of \citet{levins1968evolution} on evolution in changing environments and more recent attempts to integrate ``epigenetic inheritance systems'' (EIS) \citep{jablonka2014evolution}, developmental or ``evo-devo'' processes \citep{muller2007evodevoextendingtheevolutionarysynthesis} and ``niche construction'' effects \citep{laland2016anintroductiontonicheconstructiontheory}, emphasis has been put on the importance and impact of \emph{changes} on the course of the evolution of living organisms by natural selection. Through these works and others, environmental fluctuations have been linked to many central properties and mechanisms of evolution, among which the most discussed examples are modularity, plasticity \citep{west2005developmental}, genome size, mutation rates and evolvability.

For \citet{jablonka2014evolution}, changing environments unmask variations in the capacity of individuals to make adjustments to new conditions, therefore promoting plasticity. These authors contend that \say{For a lineage in a constantly changing environment, switching among several alternative heritable states was probably an advantage. While cells in one state survived in one set of conditions, those in other states did better in different circumstances.} (\emph{ibid.},~p.~318). In the same line of thought, continually varying or cyclic conditions might also explain the origin of EIS \citep{heard2014transgenerationalepigeneticinheritancemythsandmechanisms}, as ``epigenetic mutations'' are more reversible and occur more frequently than genetic ones. To illustrate this notion, \citet{lachmann1996inheritance} modeled the effects of oscillating variations, such as seasonal or daily cycles, on phenotypical inheritance. Their model predicts correctly that when the environmental cycle is longer than the reproductive cycle, while remaining relatively short otherwise, heritable variations produced by non-DNA inheritance systems are likely to be observed. 

In parallel to biological research, a number of studies in artificial life, especially evolutionary robotics \citep{floreano2000evolutionary}, have also investigated environmental variations, some of them explicitly defining the environment as a driving evolutionary force \citep{bredeche2012environmentdrivenopenende}.
%However and as far as we know, a significant majority of the works in that field still consider environmental changes only as a problem to be solved and nobody has systematically studied the general properties induced by environmental fluctuations over evolutionary dynamics.  %However, it is probably fair to say that, a significant majority of these previous work has either focused on aspects such as normal spatial variation, or has viewed environmental change as a problem to be solved.
Others, such as \citet{lipson2002origin}, showed a correlation between the modularity and the rate of change of external resources, while \citet{yu2007program} observed that populations exploit neutrality to cope with environmental fluctuations and can evolve a type of evolvability under two alternating objective functions. Both of these simulation works relied on genetic programming~(GP) and explicit fitness functions.

We wish to study the effects of such fluctuations in a model closer to the living world without using an explicit objective function. To that goal, we propose an open-ended experimental setup allowing us to systematically and quantitatively measure the influence of cyclic environmental fluctuations on the course of the evolution of cellular automata (CA). We show that such fluctuations lead to the emergence of processes similar to those exhibited by EIS.

The paper is organized as follows. First the general mechanisms of the HetCA model are explained. Then, the implementation of environmental fluctuations in HetCA is described. Next, we specify the computational setup used to study environmental fluctuations. This is followed by a report on the experimental results and a discussion of their implications. Finally, we propose a qualitative analysis and a conclusion.
